\begin{abstract}
We address the problem of computing semantic differences between a program and a patched version of the program. Our goal is to obtain a precise characterization of the difference between program versions, or establish their equivalence when no difference exists.

We focus on computing semantic differences in numerical programs where the values of variables have no a-priori bounds, and use abstract interpretation to compute an over-approximation of program differences. Computing differences and establishing equivalence under abstraction requires abstracting relationships between variables in the two programs. Towards that end, we first construct a \emph{correlating program} in which these relationships can be tracked, and then use a \emph{correlating abstract domain} to compute a sound approximation of these relationships. To better establish equivalence between correlated variables and precisely capture differences, our domain has to represent non-convex information. To balance precision and cost of this representation, our domain may over-approximate numerical information as long as equivalence between correlated variables is preserved.

We have implemented our approach in a tool called \sname{DIZY}, built on the \sname{LLVM} compiler infrastructure and the \sname{APRON} numerical abstract domain library, and applied it to a number of challenging real-world examples, including programs from the GNU core utilities, Mozilla Firefox and the Linux Kernel. Our evaluation shows that \sname{DIZY} often manages to establish equivalence, describes precise approximation of semantic differences when difference exists, and reports only a few false differences.
\end{abstract}
