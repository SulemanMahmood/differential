\section{Introduction} \seclabel{Intro}

%% what are we trying to say?
% 1. computing semantic diff is important
% 2. computing semantic diff is hard
% 3. existing approaches for computing semantic diff suck
% 4. our approach is great
% 5. technically, we use the following ideas:
%    - correlating program
%    - correlating abstract domains

% TODO:
% - we should have a clear problem definition somewhere
% - what is so special about a patched program?

Computing semantic difference for characterizing change in behavior due to code modification can be invaluable for the process of software development. Every change to the code requires careful consideration of questions like (i) did the patch add/remove the desired functionality? (ii) does the patch affect other, \emph{unexpected}, behaviors? (iii) which regression tests need be run? which break and why?. Answering these often consume a considerable amount of the development time, especially in legacy code where the developer of the original developer is no longer present.

Semantic differencing and program equivalence is considered a fundamental problem in computer science, receiving much attention from classical work~\cite{} and recently it has been flagged as a goal for static analysis methods~\cite{}. The applications of semantic equivalence and differencing range from contract checking, patch integration debugging and regression test generation and pruning. Recent work ~\cite{} also show the security implications of semantic differencing for exploit generation.

\para{Existing Techniques}
Existing techniques mostly offer under-approximating solutions, the prominent of which is regression testing which provides very limited assurance as to whether the behavior changes correctly as tests usually cover a fraction of program behaviors. Furthermore, running regression on large systems requires integrating the changed module into the entire system which can consume time, especially considering the actual time of running the entire regression. Godlin and Strichman~\cite{GodlinStrichman09} rely on bounded model checking techniques to produce a (binary) result regarding (input-output) equivalence two closely related numerical programs. \TODO{Further work~\cite{KawaguchiLahiriRebelo10} define the notion of conditional equivalence but are unable to describe difference in loops}. Other techniques for describing difference
~\cite{DwyerElbaumPerson08,EnglerRamos12} which rely on
symbolic execution supply unsound results as they are limited by loops and
essentially cover a subset of program behavior. \TODO{mention classical work like \cite{Horwitz:PLDI90,Horwitz:TOPLAS89}}

We present an approach based on abstract program interpretation for a \textbf{sound}, succinct representation of changed program behaviors and proving equivalence. Our method focuses on abstracting relationships between variables, and therefore behaviors, in both versions allowing us to achieve a precise description of difference and prove equivalence while ignoring other program information which may encumber a traditional analysis but is less relevant in our setting\footnote[1]{Eran: you didn't like this sentence, i love it, i'm trying to say we abstract away numerical information etc. and focus on relationships, thus we have better results since we are not encumbered by this less relevant information}.

\para{Problem Definition}
We define the problem of program equivalence and difference as follows: Given a pair of programs $(P,P')$ which agree on the number and type of inputs, for every execution of $P$ that originate from an input $i$ and a matching execution of $P'$ that originates from the \textbf{same input $i$} we describe:
\begin{itemize}
\item Whether these executions agree on output i.e. exhibit the same behavior.
\item In case of difference in behavior, provide a description of difference.
\end{itemize}
We intentionally define the notion of input and output equivalence loosely as this allows a more flexible definition of difference.

As the number of executions is unbound, we require abstraction in order to soundly compute this result. Though the notion of difference is well defined in the concrete case, defining and soundly computing it under abstraction is challenging as:
\begin{itemize}
\item Differencing requires correlation of \emph{different program executions} meaning the abstraction must limit itself to capturing differences of same-input executions, which is problematic under trace abstraction.
\item Establishing equivalence under abstraction is challenging since equality under abstraction does not entail concrete equality.
\end{itemize}

\para{Correlating Program}
Abstracting relationships allows us to maintain focus on difference while
omitting (whenever necessary for scalability) parts of the behavior that does
not entail difference. In order to monitor these relationships we created a
\emph{correlating program} which captures the behavior of both the original
program and its patched version. Instead of designing a correlating semantics
that is capable of co-executing two programs, we chose to automatically
construct the correlating program such that we can benefit from the use of
standard analysis frameworks for analyzing the resulting program. Another
advantage of this new construct, is that you may apply other methods for
equivalence checking directly on it~\cite{EnglerRamos11} as the correlation
allows for a much more fine-grained equivalence checking (between local
variables and not only output).

\para{Correlating Abstraction}
Our abstraction holds data of both sets of variables, joined together and is
initialized to hold equality over all matched variables. This means we can
reflect relationships without necessarily knowing the actual value of a
variables (we can know that $x_{old} = x_{new}$ even though actual values are
unknown). We ran our analysis over the correlating program while updated the
domain to reflect program behavior. Since some updates may result in non-convex information (e.g. taking  a condition of the form $x \neq 0$ into account), our domain has to represent non-convex information, at least temporarily. We address this by working with a powerset domain of a convex representation with partitioning according to equivalence criteria to avoid exponential blowup. Our domain may over-approximate numerical information as long as equivalence between correlated variables is preserved.

\subsection{Main Contributions}
The main contributions of this paper are as follows:
\begin{itemize}
\item we present a method for abstract interpretation of a pair of programs $(P,P')$ for \emph{sound} semantic equivalence and differencing by abstracting direct relationships between $(P,P')$ variables in a partially disjunctive domain. We describe a partitioning technique for state reduction and scaling. We define a widening operator for abstracting unbound paths in our domain.
\item we phrase a new technique for syntactically interleaving a pair of programs $(P,P')$ for the creation of a \emph{correlating program} $P \bowtie P'$ which contains the semantics of both programs. We propose an analysis over the program for characterizing program equivalence and difference, based on the aforementioned abstraction, given the properties of the correlating program which aligns $(P,P')$ executions.
\item We have implemented our approach in a tool based on the LLVM compiler infrastructure and the APRON numerical abstract
    domain library, and evaluated it using over 50 patches from open-source software including GNU core utilities, Mozilla
    Firefox, and the Linux Kernel. Our evaluation shows that the tool often manages to establish equivalence, reports useful
    approximation of semantic differences when differences exists, and reports only a few false differences.
\end{itemize}


