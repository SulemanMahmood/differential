\section{Introduction} \seclabel{Intro}

When applying a patch to a procedure, the programmer has very limited means for acquiring a description of the change the patch made to the procedure behavior. Existing techniques for proving patch equivalence will only supply the programmer with a binary answer ~\cite{GodlinStrichman} as to the (input-output) equivalence a program but no description of the difference is supplied. Further work ~\cite{KawaguchiLahiriRebelo10} allows refining the equivalence proof by providing a set of constraints under which equivalence is desired but requires the programmer to manually deduce these. Other techniques for describing difference ~\cite{DwyerElbaumPerson08,HawblitzelKawaguchiLahiriRebelo12} which rely on symbolic execution supply unsound results as they are limited by loops and essentially cover a subset of program behavior. 
We present a novel approach which allows for a sound description of difference for programs with loops. Our technique employs methods of abstract interpretation for over-approximating the difference in behaviors, by focusing the abstraction on the \emph{relationships} between program behaviors i.e. between variables values (data) and conditionals (path) in the two versions.

The idea of abstracting relationships holds great potential, as it allows us to maintain focus on difference while omitting (whenever necessary for scalability) parts of the behavior that does not entail difference. In order to monitor these relationships we created a \emph{correlating program} which captures the behavior of both the original program and its patched version. Instead of designing a correlating semantics that is capable of co-executing two programs, we chose to automatically construct the correlating program such that we can benefit from the use of standard analysis frameworks for analyzing the resulting program. Another advantage of this new construct, is that you may apply other methods for equivalence checking directly on it ~\cite{EnglerRamos11} as the correlation allows for a much more fine-grained equivalence checking (between local variables and not only output).

Another novel aspect of our work, is that we allow checking for equivalence in every point of execution, and for every variable, while previous approaches focus only on input-output equivalence. This enables detection of key differences that impact the correctness of the patch: if the changed behavior includes a bug manifested by a local variable (for instance: array index out of bounds), we will detect and describe it while previous work only detected it when propagated to the output and equivalence may have been reported although a bug was introduced. This also provides a challenge as we need to carefully choose the program locations where we check for difference otherwise we will spuriously detect difference.

Our abstraction holds data of both sets of variables, joined together and is initialized to hold equality over all matched variables. This means we can reflect relationships without necessarily knowing the actual value of a variables (we can know that $x_old = x_new$ even though actual values are unknown). We ran out analysis over the correlating program while updated the domain to reflect program behavior. However we quickly noticed that

To establish equivalence between correlated variables and precisely capture differences, our domain has to maintain correlating information even when other information is abstracted away.

Since some updates may result in non-convex information (e.g. taking  a condition of the form $x \neq 0$ into account), our domain has to represent non-convex information, at least temporarily. We address this by working with a powerset domain of a convex representation. To avoid exponential blowup, our join operator may over-approximate numerical information as long as equivalence between correlated variables is preserved.

In some cases, it would have been sufficient to use alternative domains that are capable of representing richer information, such as interval polyhedra~\cite{CMWC:SAS09}, or other numerical domains that can represent non-convex information (e.g., \cite{TODO}). The recent donut domain~\cite{GIBMG:VMCAI12} may be of particular interest for this purpose. However, the general principle of having to preserve correlating information even when information about the values is abstracted away, holds in all of these cases.

%% Talk about the analysis and the correlating domain

%% Talk about why the domain needs to be a power set

%% Talk about the way it was minimized while keeping correlation data

%% Talk about the way it was widened

%% Talk about how the delta is computed

In this paper, we present a technique based on abstract interpretation, that is able to compute an over-approximation of the difference between numerical programs or establish their equivalence when no difference exits. The approach is based on two key ideas: (i)~create a \emph{union program} that captures the behavior of both the original program and its patched version; (ii)~analyze the union program with a \emph{correlating domain} that captures relationships between values of variables in the original program and values of variables in the patched version.

The idea of a union program is similar to that of self-composition~\cite{BartheDArgenioRezk04,AikenTerauchi05}, but the way in which statements in the union program are combined is carefully designed to keep the steps of the two programs close to each other. Rather than having the patched program sequentially composed after the original program, our union program interleaves the two versions. Analysis of the union program can then recover equivalence between values of correlated variables even when equivalence is \emph{temporarily} violated by an update in one version, as the corresponding update in the other version follows shortly thereafter.

\TODO{check whether Aiken paper SAS'05 does some sort of interleaving as well}

\TODO{also need to say that there is the problem of choosing differencing points}

\DONE{there is some term they used in the security literature? self-composition? cite Nauman}

\subsection{Main Contributions} 
The main contributions of this paper are as follows:
\begin{itemize}
\item we phrase the problem of semantic differential analysis as an analysis of a union program --- a single program that represents an original program and its patched version.
\item we present an approach for analyzing differences over the union program using a correlating abstract domain. Our approach is sound --- if there is a difference at a differentiation point, we cannot miss it. However, since we over-approximate differences, our approach may report false differences due to approximation.
\item We have implemented our approach in a tool based on the LLVM compiler infrastructure and the APRON numerical abstract domain library, and evaluated it using over 50 patches from open-source software including GNU core utilities, Mozilla Firefox, and the Linux Kernel. Our evaluation shows that the tool often manages to establish equivalence, reports useful approximation of semantic differences when differences exists, and reports only a few false differences.
\end{itemize}


\TODO{mention classical work like \cite{Horwitz:PLDI90,Horwitz:TOPLAS89}} 