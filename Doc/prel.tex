\section{Preliminaries} \seclabel{Preliminaries}
We use the following standard concrete semantics definitions for a program:

\paragraph{Program Location / Label} \deflabel{ProgramLabel}
A program location $loc \in Loc$, also referred to as label denoted $lab$, is a unique identifier for a certain location in a program corresponding to the value of the program counter at a certain point in the execution of the program. We also define two special labels for the start and exit locations of the program as $begin$ and $fin$ respectively.

\paragraph{Concrete State} \deflabel{ConcreteState}
Given a set of variables $Var$, a set of possible values for these variables $Val$ and the set of locations $Loc$, a \emph{concrete program state} is a tuple $\sigma \triangleq \langle loc, values \rangle \in \Sigma$ mapping the set of program variables to their concrete value at a certain program location $loc$ i.e. $values : Var \rightarrow Val$. The set of all possible states of a program $P$ is denoted $\Sigma_{P}$.

\paragraph{Program} \deflabel{Program}
We describe an imperative program $P$, as a tuple $(Val,Var,\rightarrow,\Sigma_{0})$ where $\rightarrow : \Sigma_{P} \times \Sigma_{P} $  is a transition system which given a concrete program state returns the following state in the program and $\Sigma_{0}$ is a set of initial states of the program. Our formal semantics need not deal with errors states therefore we ignore crash states of the programs, as well as inter-procedural programs since our work deals with function calls by either ignoring them when equivalence was proven or by inlining them (we exclude recursion for now).

\paragraph{Concrete Trace} \deflabel{ConcreteTrace}
A program trace $\pi \in \Sigma^*_{P}$, is a sequence of states $\langle \sigma_0,\sigma_1,... \rangle$ describing a single execution of the program. Each of the states corresponds to a certain location in the program where the trace originated from. Every program can be described by the set of all possible traces for its run $\semp{P} \subseteq \Sigma^*$. We refer to these semantics as concrete state semantics. We also define the following standard operations on traces:
\begin{itemize}
\item $label : \Sigma_{P} \rightarrow Lab$ maps a state to the program label at which it appears.
\item $last : \Sigma_{P}^* \rightarrow \Sigma_{P}$ returns the last state in a trace.
\item $pre : \semp{P} \rightarrow 2^{\Sigma_{P}^*}$ for a trace $\pi$ is the set of all prefixes of $\pi$.
\item $states : \semp{P} \rightarrow 2^{\Sigma_{P}}$ for a trace $\pi$ is the set of actual states $\pi$ is composed of.
\end{itemize}


We shortly describe a \emph{product state} $\sigma_{\times} \in \Sigma_{P \times P'}$ as a pair of states $\langle \sigma,\sigma' \rangle$, a \emph{product program} $P \times P'$ as a product of the transition systems of the underlying programs and \emph{product trace} as a sequence of produce states.

