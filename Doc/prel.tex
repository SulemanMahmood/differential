\section{Preliminaries} \seclabel{Preliminaries}
We use the following standard concrete semantics definitions for a program:

\paragraph{Program} \deflabel{Program} A program $P$,...We are able to convert any program to this format...We exclude recursion for now.


\paragraph{Program Location} \deflabel{ProgramLocation}
A program location $loc \in Loc$,

\paragraph{Program Label} \deflabel{ProgramLabel}
A program label $lab \in Lab$, is a unique identifier for a certain location in a program. Every location has a label (usually the program counter). We also define two special labels for the start and exit locations of the program as $begin$ and $fin$ respectively.

\paragraph{Concrete State} \deflabel{ConcreteState}
A concrete program state is a tuple $\sigma \equiv \langle loc, values \rangle \in \Sigma$ mapping the set of (integer) program variables to their concrete (integer) value at a certain program location $loc$ i.e. $values : Var \rightarrow Val$. The set of all possible states of a program $P$ is denoted $\semp{P}$.

\paragraph{Concrete Trace} \deflabel{ConcreteTrace}
A program trace $\pi \in \Sigma^*$, is a sequence of states $\sigma_0,\sigma_1,...$ describing a single execution of the program. Each of the states corresponds to a certain location in the program where the trace originated from. Every program can be described by the set of all possible traces for it's run $\Pi \subseteq \Sigma^*$. We refer to these semantics as concrete state semantics. We also define the following standard operations on traces:
\begin{itemize}
\item $label \equiv \semp{P} \rightarrow Lab$ maps a state to the program label at which it appears.
\item $last \equiv \semp{P}^* \rightarrow \semp{P}$ returns the last state in a trace.
\item $pre \equiv \Pi \rightarrow 2^{\semp{P}^*}$ for a trace $\pi$ is the set of all prefixes of $\pi$.
\end{itemize}

