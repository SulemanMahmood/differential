\section{Worklist}

\subsection{Points to Hammer}

\begin{enumerate}
\item a special kind of self composition, where correlated steps are kept together. This is particularly important when handling loops.
\end{enumerate}

\subsection{TODO}
\begin{enumerate}
\DONE{\item show me an example of where comparing abstract values at the end is not sound (e.g., we had some example with intervals that demonstrates this).}
\item run on uc-klee examples.
\item Consider describing each sub-state as a single (possibly looping) path of execution in both programs that originated from the same input.
\end{enumerate}


\subsection{Questions}

\begin{enumerate}
\item how come you don't need the ``product program''?
\item what are the theorems that you provide? (no reason to have definitions if there are no theorems).
\item what abstract domains can we use as ``underlying domains'' for our abstraction? Do we have any particular requirements from the abstract domains (one requirement is being relational).
\item what makes a ``patched version of a program'' different from just saying ``a different program''? In other words - what are the requirements on the difference between $P$ and $P'$?
\item can we claim that our abstraction ``forgets'' paths along which equivalence is established, but keeps apart paths along which the is a difference, hoping that it will re-converge later?
\item (intuition only) what if we correlate badly and lose soundness? one can propose a 2 "trick" programs that correlating them our way gives a result that loses difference.
\item WHY DO WE CHECK DIFFERENCE ABOVE THE SUB-STATE LEVEL? doesn't that mean we compare different paths? isn't that bad?
\end{enumerate}

\subsection{Useful text fragments}

Differential static analysis is useful for regression debugging~\cite{TODO}, and may also lead to an automated approach for patch-based exploit generation~\cite{TODO}. Such automated exploit generation enables the organization releasing a patch to estimate the attack surface exposed by its release. Furthermore, it enables the organization releasing the patch to reduce vulnerability to manual and automated patch-based exploitation.


\begin{figure}
\begin{lstlisting}
if (input % 2 == 0) goto 2 else  goto 4;
s := input+2;
goto 5;
s := input+3;
if (s>input) goto 6 else goto ERROR;
ptr := realloc(ptr,s);
// use ptr[0], ptr[1], ... ptr[input-1]
\end{lstlisting}
\caption{Example from \cite{TODO}}
\end{figure} 