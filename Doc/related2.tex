\section{Related Work} \seclabel{Related}

Our work has been mainly inspired by recent work identifying program differencing as having vast security implications~\cite{BrumleyPoosankamSongZheng08,SongSunZhang09} as well as advancements made in the field of under-approximations of program equivalence~\cite{GodlinStrichman09, KawaguchiLahiriRebelo10, DwyerElbaumPerson08, EnglerRamos11}.

Important work in the problem of equivalence of combinatorial circuits for hardware verification~\cite{KuehlmannKrohm97,BraytonChatterjeeMishchenkoEen06, ClarkeKroening03}

We rely on classic methods of abstract interpretation~\cite{CousotCousot77} for presenting an over approximating solution for semantic differencing and equivalence. To achieve this we devised a static analysis over a newly defined construct we call a correlating program. The idea of a correlating program is similar to that of
self-composition~\cite{BartheDArgenioRezk04, AikenTerauchi05} except that we compose two different programs in a interleaving designed to maintain a close correlation between them. The use of a correlating construct for differencing is novel as previous methods mainly use sequential composition~\cite{GodlinStrichman09, KawaguchiLahiriRebelo10, DwyerElbaumPerson08, EnglerRamos11}, disregarding possible program correlation.

We base our analysis on a relational abstraction ~\cite{CousotHalbwachs78, Mine07} that allows us to reason about variables of different programs. The abstraction is further refined towards a disjunctive domain, similar to trace partitioning~\cite{MauborgneRival07} and we use an equivalence based partitioning criteria, which is apt to our purposes.

Symbolic execution based methods~\cite{DwyerElbaumPerson08, EnglerRamos11} offer practical equivalence verification techniques for loop and recursion free programs with small state space. These works complement each other in regards to reporting difference as one ~\cite{DwyerElbaumPerson08} presents an over approximating description of difference they call differential summaries and the other~\cite{EnglerRamos11} presents an under approximating description including concrete inputs for test cases demonstrating difference in behavior. An interesting question is how could these methods be combined iteratively to achieve better precision.

Bounded model checking based work~\cite{GodlinStrichman09} present the notion of partial equivalence which allows checking for equivalence under specific consitions, supplied by the user.

\cite{JacksonLadd94}

\TODO{ \cite{GS:DAC09} }


\paragraph{Determining corresponding components}

As suggested in \cite{Horwitz:PLDI90}, one possibility is to rely on the editing sequence that creates the new version from the original one. Another option is using various syntactic differencing algorithms as a base for computing correspondence tags.

\TODO{their idea for computing correspondence, is to minimize the ``size of change''. They have two different notions of size of change.}

\cite{ARRSY:CAV07} introduced a correlating heap semantics for verifying linearizability of concurrent programs. In their work, a correlating heap semantics is used to establish correspondence between a concurrent program and a sequential version of the program at specific linearization points.  